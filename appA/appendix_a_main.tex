\chapter{Media Content}

If the dissertation has a DVD or pendrive attached to it, you will need a section which explains what is on the media (structure, files, data, etc.).  This could be a table with filename and description.

\blindtext[5]

\section*{Full list of machine configuratoin}
"Retardation";""
"Relaxation";""
"Arbeitsdifferenz zw. Be- u. Entlastung";""
"Elastische Dehnung";""
"Gesamtdehnung nach Wartezeit";""
"Gesamtdehnung vor Wartezeit";"1:2,48 mm"
"Restdehnung";""
"1. Stufe";1;"mm"
"Querschnitt";10;"mm"
"Bruchdehnung";"nicht bekannt"
"Dehnung bei Fmax";2,3194;"mm"
"andere Geschwindigkeit für Entlastung";300;"mm/min"
"Ansteuerbare Probenhalter aktiv";"Nein"
"Anzahl Stufen";4;" "
"Anzahl Zyklen";1;" "
"Auf L=0 mm fahren?";"Nein"
"Au�endurchmesser";10;"mm"
"Automatische Kraftnullung";"Nein"
"Beginn E-Modulermittlung";0,05;"%"
"Bezeichnung der Unterserie";"0"
"Bezugswert während Wartezeit oben";""
"Bezugswert w�hrend Wartezeit unten";""
"Breite der Streifenprobe b0";20;"mm"
"Dehnungsverhältnis";""
"Datum";"28.11.2022"
"Dichte der Messprobe";1;"g/cm2"
"Dummy Aufnehmerumschaltung";"Keine Umschaltung"
"Durchmesser d0";1;"mm"
"Eingabeaufforderung vor Prüfung";"Zwingend"
"Einspannlänge";2,97009;"mm"
"Einspannlänge nach Vorlaufweg";100;"mm"
"E-Modul";"nicht bekannt"
"Ende E-Modulermittlung";0,25;"%"
"Erholungsdauer";15;"min"
"Erwartete Steifigkeit der Probe";210;"kN/mm�"
"Obere Kraft nach Wartezeit";""
"Obere Kraft vor Wartezeit";"1:40,06 N"
"Faktor Dehnungsumrechnung";0;" "
"Faktor Spannungsumrechnung";0;" "
"Feinheit";10;"dtex"
"Fl�chengewicht der Messprobe";80;"g/m2"
"Geregelt Positionieren ?";"Nein"
"Geschwindigkeit E-Modul";1;"mm/min"
"Geschwindigkeit Entlastung";50;"mm/min"
"Geschwindigkeit für Weg nach Bruch";500;"mm/min"
"Geschwindigkeit Vorlaufweg";100;"mm/min"
"Geschwindigkeit Zyklus";80;"mm/min"
"Gewicht der Probe";1;"g"
"Halteart an den Stufen";"lagegeregelt"
"Halteart bei Vorkraft";"lagegeregelt"
"Halteart oben";"kraftgeregelt"
"Halteart unten";"lagegeregelt"
"Haltezeit bei Vorkraft";0;"s"
"Kraft bei E-Modul Ende";"nicht bekannt"
"Höhe der Zusatzwerkzeuge";0;"mm"
"I Anteil Dehnungsgeschwindigkeitsregler";0;" "
"I Anteil Kraftgeschwindigkeitsregler";0;" "
"Innendurchmesser";8;"mm"
"Keilspannfaktor";5;"bar/kN"
"Kein erneutes Anfahren der Vorkraft";"Nein"
"Keine Konsistenzprüfung durchführen";"Ja"
"Kraft nullen nach Vorkraft";"Nein"
"Kraftabschaltschwelle";60;"%Fmax"
"Kraftaufnehmer";""
"Kraftkonstanthaltung beim Spannvorgang";"Nein"
"Kraftschwelle f�r Bruchuntersuchung";0,5;"%Fnom"
"Kraftsprung";5;"%"
"Kreisverstärkung Dehnungsregelung";1;" "
"Kreisverstärkung Kraftregelung";1;" "
"Kunde";""
"Messlänge";40;"mm"
"Länge der Probe";1;"mm"
"LE - Abweichung";0,5;"mm"
"LE einstellen nach Pr�fung";"Ja"
"LE Übernehmen";"Nein"
"LE-Geschwindigkeit";100;"mm/min"
"Kraft bei E-Modul Beginn";"nicht bekannt"
"Maschinendaten";"1454MO WN:805754
Traversenwegaufnehmer WN:805754
Kraftsensor ID:0 WN:820243 20 kN
"
"Material";"Stahl"
"max. Längenänderung";0;"mm"
"max. zeitlicher Kraftabfall";1;"N"
"Max. zulässige Kraft am Prüfungsende";300;"N"
"maximale Versuchsdauer";1;"min"
"Messlänge korrigieren";"Nein"
"Messlänge Standardweg";50;"mm"
"Nachspannen";"nie"
"Nachspann-Intervall";1;"s"
"Negativer Dehnungssprung";10;"%"
"negativer Querschnittskorrekturwert";0;"mm�"
"Nur bestimmte Zyklen erfassen?";""
"obere Kraftgrenze";50;"N"
"Oberer Umkehrpunkt";17,5;"mm"
"�ffnungszeit";0;"s"
"P Anteil Dehnungsgeschwindigkeitsregler";0;" "
"P Anteil Kraftgeschwindigkeitsregler";0;" "
"Parallele Probenl�nge";40;"mm"
"Positiver Dehnungssprung";10;"%"
"Probe entlasten";"Nein"
"Probenbreite b0";20;"mm"
"Probendicke a0";0,5;"mm"
"Probenform f�r Querschnittsberechnung";"Flachprobe"
"Probenhalter";""
"Probenhalter nach Pr�fung �ffnen";"Ja"
"Probennummer";62;" "
"Probennummer in Unterserie";0;""
"Prüfart";"Druck"
"Prüfer";"Kurrle"
"Prüfgeschwindigkeit";5;"mm/min"
"Prüfnorm";"DIN 178"
"Prüfplatzname";""
"Prüfverfahren";"stufige Belastung"
"Querschnittseingabe";1;"mm�"
"Querschnittskorrekturfaktor";1;" "
"Bruchkraft";"nicht bekannt"
"Regelgeschwindigkeit f�r Kraftkonstanthaltung";3;"mm/min"
"Reglerüberwachung";"Nein"
"Relaxation/Retardation?";"Relaxation"
"Kraftmaximum";40,4454;"N"
"Rohrdefinition";"Innendurchmesser"
"Schließdruck";10;"bar"
"Solltemperatur";0;"C"
"Spanndruck";100;"bar"
"Steifigkeit der Pr�fanordnung";5000;"N/mm"
"Steuerung der Umkehrpunkte";"geregelte Positionierung"
"Stichprobengröße";0;" "
"Toleranzen";""
"TRS - Kraftintervall";1;"N"
"TRS - Wegintervall bis Bruch";0,1;"mm"
"TRS - Zeitintervall";0,1;"s"
"TRS-Bereich für Bruchuntersuchung";50;" "
"Uhrzeit";"03:51:38"
"Unterer Umkehrpunkt";0;"mm"
"Unterseriennummer";0;""
"Verzögerung bei Geschwindigkeitsumschaltung";100;" "
"Vorgabe für Bezugswert w�hrend Wartezeit oben";0;"s"
"Vorgabe für Bezugswert w�hrend Wartezeit unten";0;"s"
"Vorgabe für die Messl�nge der Traverse";"Parallele Probenl�nge"
"Vorkraft";1;"N"
"Vorkraft erst nach Solltemperatur starten";"Ja"
"Vorkraft Haltezeitmodus";"keine Haltezeit"
"Vorkraft-Geschwindigkeit";100;"mm/min"
"Vorkraftüberwachung im Vorlaufweg";"Ja"
"Arbeit im Belastungsteil";""
"Arbeit im Entlastungsteil";""
"Wandstärke";2;"mm"
"Wartezeit an den Stufen";1;"s"
"Wartezeit nach Solltemperatur";15;"s"
"Wartezeit oben";15;"s"
"Wartezeit unten";15;"s"
"Weg nach Bruch";5;"mm"
"Wegaufnehmer";""
"Zeit bis Vorkraft";60;"s"
"Zeitintervall für Kraftabfall";1;"s"
"Zug- / Druckversuch nach den Zyklen";"Nein"
"Zulässige Abweichung für Kraftnullpunkt";0,5;"%Fnom"
"Zulässige Kraftabweichung für Kraftnullpunkt";0,5;"%Fnom"
"Zulässige Temperaturabweichung";1;"C"
"Zunahme je Stufe";1;"mm"
"Zunahme je Zyklus";1;"mm"
"Zwingende Eingaben vor Prüfung";"Ja"
