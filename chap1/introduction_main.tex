\chapter{Introduction}

Sheet metal forming has been used for centuries in different manufacturing industries to create a wide range of products for different applications. 
Sheet metal bending and stamping can be considered as the most important variants in the forming industry. \cite[p. 1]{cruz_applicationmachinelearning_2021} 
Therefore these have been continuously improved in recent decades to meet the growing demand especially in  automotive and aircraft industries with the goal to reduce energy efficiency and emissions. \cite[p. 4]{zheng_reviewformingtechniques_2018}

Sprinback is a common phenomenon in sheet metal forming processes. It is a deformation of the sheet metal that occurs when the sheet metal is bent. Therefore, predicting the spring back is important to reduce the number of trial and error cycles in the manufacturing process. \cite[p. 1]{cruz_applicationmachinelearning_2021} 
Sheet metal forming is a complex process that involves a large number of variables and parameters Therefore, it is difficult to predict the spring back accurately, which makes it an interesting case for machine learning.

In order to predict springback with minimium errors, this thesis build and evaluates different machine learning models to predict the springback of a sheet metal. The models are evaluated based on the mean absolute error (MAE) and the root mean squared error (RMSE). The best model is then used to predict the springback of a sheet metal with different parameters. 

\section{Problem statement}
With the rapid development in manufacturing and increasing demand for high-quality products, the 
requirement to produce parts with high precision and accuracy has become a need of the manufacturing 
world, since it is known that springback is an undesired outcome, so the need for minimizing this in 
sheet metal parts is of utmost importance, this could not be achieved by adopting traditional approaches 
to predicting springback, so a newer approach known as a machine learning approach is adopted. 
Machine learning is a branch of Artificial intelligence in which, given input data points and output 
value, a computer algorithm learns rules by Analyzing the data. In other words, it gives systems the 
ability to learn and improve themselves without explicitly being programmed. The recent advancement 
in technology and the development of manufacturing 4.0 also triggered the need for machine learning. 
It means that the machines are producing data at an unprecedented scale, so now it is needed to have 
fast learning algorithms that can give accurate results in a short amount of time. 
\cite[]{baig_machinelearningprediction_2021}
