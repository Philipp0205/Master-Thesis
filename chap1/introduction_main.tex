\chapter{Introduction}

Sheet metal forming has been used for centuries in different manufacturing industries
to create a wide range of products for different applications.
Sheet metal bending and stamping can be considered as the most important variants in
the forming industry. \cite[p. 1]{cruz_applicationmachinelearning_2021}
Therefore these have been continuously improved in recent decades to meet the growing
demand especially in automotive and aircraft industries with the goal to reduce energy
efficiency and emissions. \cite[p. 4]{zheng_reviewformingtechniques_2018}

Sprinback is a common phenomenon in sheet metal forming processes. It is a deformation
of the sheet metal that occurs when the sheet metal is bent. Therefore, predicting the
spring back is important to reduce the number of trial and error cycles in the
manufacturing process. \cite[p. 1]{cruz_applicationmachinelearning_2021}
Sheet metal forming is a complex process that involves a large number of variables and
parameters Therefore, it is difficult to predict the spring back accurately, which
makes it an interesting case for machine learning.

In order to predict springback with minimium errors, this thesis build and evaluates
different machine learning models to predict the springback of a sheet metal. The
models are evaluated based on the mean absolute error (MAE) and the root mean squared
error (RMSE). The best model is then used to predict the springback of a sheet metal
with different parameters.




%"There are two major types of supervised machine learning problems,
%called classification and regression." \cite[p. 34]{muller_introductionmachine_2016}
%
%"For regression tasks, the goal is to predict a continuous number, or a
%floating-point number in programming terms (or real number in
%mathematical terms). Predicting a person’s annual income from their
%education, their age, and where they live is an example of a regression
%task. When predicting income, the predicted value is an amount, and
%can be any number in a given range. Another example of a regression task
%is predicting the yield of a corn farm given attributes such as
%previous yields, weather, and number of employees working on the farm.
%The yield again can be an arbitrary number." \cite[p. 34]{
%    muller_introductionmachinelearning_2016}