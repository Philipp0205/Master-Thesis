\chapter{Introduction}\label{ch:introduction}
% Ausgangssituation & Themendarstellung

% Relevanz des Themas
Sheet metal forming has been an integral part of manufacturing industries for centuries, creating
diverse range of products used in different applications.
From automobiles to airplanes, sheet metal bending and stamping are the go-to methods for
crafting intricate parts and components~\cite[p. 1]{cruz_applicationmachinelearning_2021}.
But, as the demand for greater efficiency and reduced emissions grows, the need to improve these
processes becomes even more crucial~\cite[p. 4]{zheng_reviewformingtechniques_2018}.


\section{Problem Identification and Motivation}\label{sec:problem-identification-and-motivation}
Over the past several decades, extensive research has been conducted on sheet metal forming technologies because of
to satisfy the growing demand for lightweight metal parts.
One common challenge in the field is the phenomenon of spring back, which is the tendency of a sheet metal to return
to its original shape after being bent.
This deformation can significantly disrupt the entire manufacturing process, leading to increased trial and error
cycles~\cite[p. 1]{cruz_applicationmachinelearning_2021}.
The complexity of sheet metal forming arises from the non-linear behavior caused by deformations of the metal sheet,
making spring back prediction particularly difficult.
Consequently, researchers have explored various compensation methods, including the use of Machine Learning (ML)
models~\cite[p.1]{liu_newmachinelearning_2012}.

Traditional approaches often rely on trial and error techniques~\cite[p. 1]{dib_singleensembleclassifiers_2020}.
Expert consultations in the field of sheet metal forming revealed that a common trial and error approach involves the
creation of `technology tables' which contain bending parameters and resulting spring back data
(see appendix~\ref{sec:technology-tables}).
These tables are generated through numerous experiments with varying bending angles and metal sheets.
However, this method is both time-consuming and costly, making it ill-suited for producing high-volume and low-cost
components (personal communication, Dr. Wolfram Hochstrate and M. Sc. Peter Lange).

These challenges underscore the need for an accurate ML model to predict spring back.
Current research predominantly focuses on applying ML to datasets generated via simulation and other bending
techniques (see section~\ref{sec:state-of-research} `State of Research').
However, there has been limited attention given to the utilization of real-world data and the air-bending method.
This study aims to fill these gaps by exclusively using air-bending to generate a comprehensive dataset, thus
enhancing the relevance and motivation for applying ML to real-world scenarios.

The primary goal of this project is to accurately predict spring back and develop \ac{ML} models that satisfy crucial
quality attributes.
To evaluate the performance of the ML models, six key Design Principles (DPs) have been identified:
correctness,relevance, robustness, stability, resource utilization, and interpretability
(see~\ref{sec:objectives-of-a-solution}).
The objective is to create models that excel in each of these areas, ensuring the predictions are as precise and
reliable as possible.

This problem identification leads to the following research questions:

\begin{tcolorbox}[arc=0pt,boxrule=0.5pt]
    \textbf{Hypothesis 1:} Machine Learning (ML) models can accurately predict spring back in sheet metal forming using
    real -world air-bending data, outperforming traditional trial-and-error methods.

    \textbf{Hypothesis 2:} Specific ML models or combinations of models (e.g., ensemble methods) will yield better
    performance
    in terms of the six Design Principles (DPs) compared to other models.

    \textbf{Hypothesis 3:} ML models with high interpretability can provide valuable insights into the factors that
    contribute to spring back, potentially leading to better understanding of the underlying physical processes and
    more effective compensation strategies.
\end{tcolorbox}


\section{Research Method and Structure}\label{sec:research-method-and-structure}
% Methode
The research methodology employed in this thesis is Design Science Research (DSR), which is a
research approach that emphasizes the development and assessment of innovative solutions to real-
world problems.
DSR aims to generate new knowledge by creating, implementing, and evaluating new artifacts or
solutions.

It is worth taking a look at figure~\ref{fig:dsr_process} up front to get a understanding on how this study is
structured.
Some changes have been made to the DSR process which are reasoned in the chapter~\ref{ch:research-methodology}.
The first activity `Identify Problem \& `Motivation' is already covered in this introduction and the order of the
activities 4 and 5 where changed to suit the structure of this study.

% Aufbau der Bachelorarbeit
The thesis follows a two-phase process of ``build'' and ``evaluate'' within the DSR approach.
The research methodology chapter (\ref{ch:research-methodology}) elaborates on how this approach is
utilized throughout the thesis.

