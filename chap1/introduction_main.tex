\chapter{Introduction}

Sheet metal forming has been used for centuries in different manufacturing industries to create a wide range of products for different applications. 
Sheet metal bending and stamping can be considered as the most important variants in the forming industry. \cite[p. 1]{cruz_applicationmachinelearning_2021} 
Therefore these have been continuously improved in recent decades to meet the growing demand especially in  automotive and aircraft industries with the goal to reduce energy efficiency and emissions. \cite[p. 4]{zheng_reviewformingtechniques_2018}

Sprinback is a common phenomenon in sheet metal forming processes. It is a deformation of the sheet metal that occurs when the sheet metal is bent. Therefore, predicting the spring back is important to reduce the number of trial and error cycles in the manufacturing process. \cite[p. 1]{cruz_applicationmachinelearning_2021} 
Sheet metal forming is a complex process that involves a large number of variables and parameters Therefore, it is difficult to predict the spring back accurately, which makes it an interesting case for machine learning.

In order to predict springback with minimium errors, this thesis build and evaluates different machine learning models to predict the springback of a sheet metal. The models are evaluated based on the mean absolute error (MAE) and the root mean squared error (RMSE). The best model is then used to predict the springback of a sheet metal with different parameters. 

% Sheet metal forming is a manufacturing process that is
% commonly used for producing high-volume and low-cost
% components in the automotive, aircraft and home appliance
% industries. In this process, forces are applied to the metallic
% sheet to modify its geometry, enabling the production of
% complex shapes. The forces are applied by tools whose
% geometry dictates the shape of the component. The process
% design is complex because only the final shape of the
% component is known. Moreover, the process is highly
% nonlinear due to the large deformations imposed to the
% metal sheet, which presents plastic behaviour, but also as a
% result of the evolutionary boundary conditions imposed by
% the contact between the tools and the sheet. The conven-
% tional process design is based on empirical knowledge and
% an experimental ‘‘trial-and-error’’ approach. In 
% \cite[]{dib_singleensembleclassifiers_2020}