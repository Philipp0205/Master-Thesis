\chapter{Introduction}

“Sheet metal forming has been employed for centuries in diverse manufacturing industries to create a wide range of products that may be used in several applications. Among different forming techniques, sheet bending and stamping can be considered the most important variants in forming industry. These techniques have been continuously improved in recent decades to meet the growing need for lightweight metallic components in the automotive sector in order to address environmental concerns about energy efficiency and emissions [1,2].” 
\cite[p. 1]{cruz_applicationmachinelearning_2021} \\

The increasing availability of data, which becomes a continually increasing trend in
multiple fields of application, has given machine learning approaches a renewed interest in recent
years. Accordingly, manufacturing processes and sheet metal forming follow such directions, having
in mind the efficiency and control of the many parameters involved, in processing and material
characterization. In this article, two applications are considered to explore the capability of machine
learning modeling through shallow artificial neural networks (ANN). 
\cite[]{cruz_applicationmachinelearning_2021}

To bend a sheet metal material, different methods can be used such as
air bending, coining, and bottom bending. Air bending, Figure 1a, is a process in which the
punch deforms the sheet by bending without the sheet being coined against the bottom
die.
. Therefore, it is frequently the preferred bending method because it provides a high
level of flexibility, as it is possible to obtain different bending angles using the same set
of tools by only controlling the punch stroke. However, this process is characterized by
strong nonlinear behavior, considering its parameters and their interrelationships [3].
In bending operations, one of the most important issues to consider is the spring-
back effect. In fact, the removal of the tools causes the release of the installed residual
stresses, leading to elastic recovery of the material and a change in the final bending angle.
Consequently, estimating the springback effect becomes a vital requirement for achieving
an accurate and regulated procedure. To address this issue, several authors tried to esti-
mate the springback behavior in bending operations in order to develop compensation
methods based on experimental, analytic and numerical approaches. 
\cite[]{cruz_applicationmachinelearning_2021}


"Sheet metal bending is a typical operation and springback is an unintended consequence of this 
operation. Since it causes fitting issues in the assembly, which leads to quality problems, anticipating it 
long before the bending operation is done is essential in today's production, so that machining 
parameters can be adjusted accordingly. In order to predict springback with minimum errors, this paper 
presents the idea for the development of machine learning models using tree-based learning algorithms 
(A class of machine learning algorithms).
"

\section{Motivation} % why is this a non trivial problem

\section{Problem statement}
With the rapid development in manufacturing and increasing demand for high-quality products, the 
requirement to produce parts with high precision and accuracy has become a need of the manufacturing 
world, since it is known that springback is an undesired outcome, so the need for minimizing this in 
sheet metal parts is of utmost importance, this could not be achieved by adopting traditional approaches 
to predicting springback, so a newer approach known as a machine learning approach is adopted. 
Machine learning is a branch of Artificial intelligence in which, given input data points and output 
value, a computer algorithm learns rules by Analyzing the data. In other words, it gives systems the 
ability to learn and improve themselves without explicitly being programmed. The recent advancement 
in technology and the development of manufacturing 4.0 also triggered the need for machine learning. 
It means that the machines are producing data at an unprecedented scale, so now it is needed to have 
fast learning algorithms that can give accurate results in a short amount of time. 
\cite[]{baig_machinelearningprediction_2021}

\section{Our Approach}

\section{Document Structure}
