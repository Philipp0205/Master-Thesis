\chapter{Introduction}\label{ch:introduction}
% Ausgangssituation & Themendarstellung

% Relevanz des Themas
Sheet metal forming has been an integral part of manufacturing industries for centuries, creating
diverse range of products used in different applications.
From automobiles to airplanes, sheet metal bending and stamping are the go-to methods for
crafting intricate parts and components~\cite[p. 1]{cruz_applicationmachinelearning_2021}.
But, as the demand for greater efficiency and reduced emissions grows, the need to improve these
processes becomes even more crucial~\cite[p. 4]{zheng_reviewformingtechniques_2018}.

% Problembeschreibung und thematische Abgrenzung
Spring back is a common phenomenon that occurs when sheet metal is bent.
It is a deformation that can derail the entire manufacturing process and lead to more trial and
error cycles~\cite[p. 1]{cruz_applicationmachinelearning_2021}.
Sheet metal forming is a complex process that involves a large number of variables and
parameters Therefore, it is difficult to predict the spring back accurately, which
makes it an interesting case for machine learning.

% Relevance & Motivation
The current focus of research is on the application of \ac{ML} to datasets generated via
simulation and other bending techniques.
However, limited attention has been paid to the use of real-world data and the air-bending method.
As a result, this study aims to address these gaps by exclusively utilizing a three-point bending
machine to generate a complete dataset.
This research aims to enhance the relevance and motivation for applying \ac{ML} to real-world
scenarios using air bending.

% Goal
The objective of this project is to accurately predict spring back and develop machine learning
models that satisfy essential quality attributes.
To assess the performance of the \ac{ML} models, six key quality attributes have been identified
: correctness, relevance, robustness, stability
, interpretability, and resource utilization.
The aim is to create models that perform well in each of these areas and ensure that the predictions
are as precise as possible.

% Methode
The research methodology employed in this thesis is Design Science Research (DSR), which is a
research approach that emphasizes the development and assessment of innovative solutions to real-
world problems.
DSR aims to generate new knowledge by creating, implementing, and evaluating new artifacts or
solutions.

% Aufbau der Bachelorarbeit
The thesis follows a two-phase process of ``build'' and ``evaluate'' within the DSR approach.
The research methodology chapter (\ref{ch:research-methodology}) elaborates on how this approach is
utilized throughout the thesis.