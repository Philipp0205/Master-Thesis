\chapter{Theoreitcal foundations}

\section{Sheet Metal Bending}
Sheet metal 
"The use of sheet metal has tremendously increased due to its application in domestic and commercial appliances. By definition, sheet metal is any form of metal that has a relatively large length to thickness ratio, their thickness typically ranges from 0.4 mm to 6 mm [1]. Sheet-metals parts are formed for different kinds of applications and are now required in almost all types of equipment, whether domestic or commercial. Automobile original equipment manufacturers (OEMs) around the globe use hightensile strength steel sheets because it provides increased strength-to-weight ratio and corresponding toughness to vehicles. Due to this reason, it is primarily used in structural parts as it provides safety and improved fuel efficiency [2]. Parts made of sheet metals are manufactured by various processes like piercing, drawing , shaping, bending, etc. Among these, the most applied manufacturing process on sheet metal is bending [3]." 
\cite{baig_machinelearningprediction_2021} 

"Bending is forming operation in which the sheet metal is forced to acquire the shapes of the cavity formed between the punch and the die. The load applied is beyond its yield strength but below its ultimate tensile strength, such that a permanent deformation is made. During this process, the metal on
the outside of the neutral plane is stretched, while the metal on the inside of the neutral plane is compressed, as shown in Fig 1." 
\cite{baig_machinelearningprediction_2021} 

% Figure " Fig 1 Compression and tensile elongation of the metal in bending"

With the rapid development in manufacturing and increasing demand for high-quality products, the requirement to produce parts with high precision and accuracy has become a need of the manufacturing world, since it is known that springback is an undesired outcome, so the need for minimizing this in sheet metal parts is of utmost importance, this could not be achieved by adopting traditional approaches to predicting springback, so a newer approach known as a machine learning approach is adopted. 
\cite{baig_machinelearningprediction_2021} 

\section{Sheets and Cuboids}
"Let’s start by assuming you wanted to build a 90° bracket out of an infinitesimally thin sheet of material, or to be practical, a piece of paper. Because it’s so thin, it actually does not contain any material, so it will bend without material deformations. To make it even simpler, we choose a bend radius of 0, which makes it a crease. In this theoretical case, the length L of the strip we need to cut out will be the sum of the two sides of the bracket, A and B."
\cite{by_artsciencebending_2016a} 

"If we now add a bend radius, our bracket will not consist of two straight sides A and B anymore, but by two shortened legs, which I will call a and b. The legs are connected by an arc of length c. So far, so good."
To think about bending a sheet of metal that has appreciable thickness, focus on an imaginary central sheet, the so-called neutral line or neutral axis, within the thickness. This neutral line behaves just like the thin sheet above, remaining undeformed during bending. The only two things we have to bear in mind mind are that the material thickness t offsets the bend radius r’ of the neutral line by half the material thickness, and our legs a and b get a bit shorter. Real-world materials like steel and aluminum do not behave exactly like this central line, but the concept of the neutral line is still useful to describe them.

\section{Bending allowance and k-factor}
"As always, real-world materials do not behave as simply as our models. After the material has taken on its new shape in between the hardened steel tools of the press, this central neutral line will be pretty messed up by the interaction. We can’t really know the course of the neutral line after the bend without a detailed and rather complex model of the material characteristics. To make things easy, an imaginary neutral line based on a simplified approximation can be used to predict the length of the flat pattern:"
"To do this, a correction factor, k, is introduced. The factor offsets the neutral line piece in the bend region from its center path until it has the length of the corresponding region of the flat pattern. The k-factor is empirically determined for a given material, material thickness, bend radius, and bending method. It reflects all real but unknown distortions in the bend region."

"Since the k-factor depends on several factors, tables of empirically determined k-factors for given setups are used. Using the k-factor, we can now calculate the bend allowance "BA", which is the length of flat material that goes into the bend region. It’s simply the arc length of the "imaginary" neutral line piece, that has been offset by the k-factor:"
"Of course, the approximation is only as realistic as the k-factor used, and it makes sense to keep your own table with k-values for the materials you intend to work with. However, the following values are a good starting point:"

\section{Springback} 
As illustrated in Fig 2, the punch (male part) having a concave shape pushes the sheet metal into the die 
(female part) having a convex shape. It causes the metal to acquire the shape of the die. When the punch 
is pulled back from the sheet metal, it tends to recover due to its elastic behaviour. This elastic recovery 
is termed as springback, as illustrated in Fig 3. Depending on the tensile nature of the material, 
springback can either be positive (i.e., when sheet metal recovers outward) or negative (i.e., when the 
sheet metal recovers inward) 
\cite{baig_machinelearningprediction_2021}

\section{Bend deduction}
"In practice, the flat pattern length is always shorter than the sum of A and B, so everything above can be condensed in the difference between A + B and L, which is called the bend deduction "BD"." 
\cite{by_artsciencebending_2016a}

“Die beim Biegevorgang stattfindende plastische Formänderung beschränkt sich dabei nicht nur auf eine reine Richtungsänderung, sondern es tritt gleichfalls eine plastische Änderung der Länge auf. So wird die dem Werkzeug zugewandte Seite des Biegeteils gestaucht, während die gegenüberliegende Seite eine Verlängerung infolge Dehnung erfährt. Dieses Verhalten während des Umformprozesses wird als Biegeverkürzung oder auch als Biegezugabe bezeichnet, je nachdem, welche Seite des Biegeteils man betrachtet.” 
\cite{rockhausen_integrationsolidworksprozesskette_2010} 

“Dabei ist diese plastische Verformung keineswegs linear und ihre Berechnung nicht trivial. Die Biegezugabe stellt einen Zahlenwert dar, der von mehreren Faktoren abhängig ist, so zum Beispiel vom Material, von der Blechdicke und den verwendeten Werkzeugen. Zwar gibt es hierfür Formeln zu ihrer Berechnung, so zum Beispiel nach DIN 6935, doch auch diese approximieren nur die in der Fertigung tatsächlich auftretenden Biegezugaben. Daher werden oft Erfahrungswerte zugrunde gelegt, die oftmals die zuverlässigere Annäherung darstellen.” 
\cite{rockhausen_integrationsolidworksprozesskette_2010} 

springback is entirely intercorrelated with the stress distribution on sheet metal as residual stresses [42]. Its behavior is also affected by material properties such as strain hardening, elastic property evolution, the presence of Bauschinger effects, elastic and plastic anisotropy, and tribology between contacting surfaces [43]. Although there are mathematical models for predicting springback in bending situations, most of them are simplistic and do not take into account all influential factors.” (Cruz et al., 2021, p. 4) 

\section{}{Machine Learning}
"Machine learning is a branch of Artificial intelligence in which, given input data points and output value, a computer algorithm learns rules by Analyzing the data. In other words, it gives systems the ability to learn and improve themselves without explicitly being programmed. The recent advancement in technology and the development of manufacturing 4.0 also triggered the need for machine learning. It means that the machines are producing data at an unprecedented scale, so now it is needed to have fast learning algorithms that can give accurate results in a short amount of time. This need triggered engineers worldwide to build new sets of algorithms that are fast at learning and can also give reliable answers. One such group of algorithms already exist which are known as tree-based learning algorithms. A tree-based learning algorithm is a group of machine learning algorithms that are used for supervised learning."
\cite{baig_machinelearningprediction_2021}


\section{Unsupervised and Supervised Learning}
\section{Regression}

\paragraph{Regression Genereal}

\begin{itemize}
    \item Explain influence of a set of variables on the outcome of another variable 
    \item Find out which input variables are the most significant influencers of the output variable 
    \item Identify a function that explains and predicts the value of the output variable when given the values of the input variables, therefore also called function fitting: y = f(x)
    \item source: regression lecture 
\end{itemize}

Types of regression
\begin{itemize}
    \item Linear Regression
    \item Logistic Regression
\end{itemize}

Key assumption linear relationship between input and outcome variables, but input and outcome variables can be transformed to achieve a linear relationship 

Regression is probabilistic, not deterministic 
– Probabilistic: Provides only expected values, based on probabilities, will include random errors 
– Deterministic: If input variables are known, output variables can be precisely determined. Examples: Newton’s Laws in Physical sciences (Foundation of classical mechanics)

\paragraph{Model Description}
...

\paragraph{Model Evaluation} 
Three aspects should be investigated in model evaluation of regression models: 
– Overall accuracy and explanatory power of the model 
– Significance of each independent variable for the outcome 
– Confirmation of the assumptions of linear regression models

- Accuarcy: (Mean) squared error 
- Explanatory power (Squared correlation)

% more in lecture 

\section{}{ML in Bending}
“Recently, there has been an increasing use of machine learning (ML) algorithms in various applications related to sheet metal forming to improve decision making and achieve cost-effective, defect-free, and optimal manufacturing quality [17,18]. The ML algorithms can be divided mainly in three categories: supervised learning, unsupervised learning [19], and reinforcement learning [20]. Generally, supervised learning is preferred and is used in classification or regression problems, encompassing support vector machine (SVM) algorithms, naive Bayes classifier, decision tree, the K-nearest neighbor (KNN) algorithm and artificial neural networks (ANN).” 
\cite{cruz_applicationmachinelearning_2021}

The authors of [21] used SVM to estimate the springback of a micro W-bending process with high prediction accuracy and generalization performance. The authors of [22] compared the performance of different machine learning algorithms (multilayer perceptron type ANN, random forest, decision tree, naive Bayes, SVM, KNN, and logistic regression) in predicting springback and maximum thinning in two different forming geometries, namely U-channel and square cup. The authors concluded that the multilayer perceptron algorithm was the best in identifying the springback, with a slightly higher score than SVM.
\cite{cruz_applicationmachinelearning_2021}


"The literature review shows that FEM methods and Machine Learning approaches are the two techniques that are vastly applied to predict the springback in sheet metals. Since FEM is slow so it cannot be used as an on-line tool in the production line for predicting springback [29]. In machine learning, most of the earlier attempts used artificial neural networks (ANN) to predict springback, which has several limitations. Using ANN, the predictions cannot be justified easily, i.e., the explainability of the answer from the neural network is very low. Neural networks require a lot of computation power to train the model. A neural network needs a large amount of data so that the model trained is generalized, rather than overfitted or under fitted to the data."
\cite{baig_machinelearningprediction_2021}

"Hence, this research article used tree-based learning algorithms which have high explainability, needs less computational power, and need less data to train the model." 
\cite{baig_machinelearningprediction_2021}




