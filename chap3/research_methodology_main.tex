\chapter{Research methodology}\label{ch:research-methodology}

The research methodology employed in this thesis adheres to the Design Science Research (DSR) approach, as detailed
in
~\cite[p. 17]{von2015management}.
DSR is a research paradigm where the researcher, acting as a designer, strives to develop artifacts that address
specific problems or questions
~\cite[p. 10]{hevner_designscienceresearch_2010}.
In the context of DSR, ``design'' is defined by Peffers et al. (2007) as the process of crafting an applicable solution
for a given problem~\cite[p. 47]{peffers2007design}.

As a methodological framework, the design-oriented research approach is particularly well-suited for addressing the
research questions at hand.
The prediction of spring back and bend deduction constitutes a significant issue in practical business applications,
as previously discussed.
Moreover, the development and implementation of machine learning models are inherently design-oriented activities.
% Artifact definition
The term ``artifact'' is deliberately broad, encompassing various forms.
In this study, the artifacts refer to \ac{ML} models that are applied to the problem of predicting spring back.
Therefore the term artifact and machine learning model are used interchangeable in this study.

DSR can be executed through diverse means, with a notable example provided by~\cite{peffers2007design},
as illustrated in Figure~\ref{fig:dsr_process}.
This approach consists of six steps, which are further divided into the overarching phases of ``Build'' and
``Evaluate''.
This thesis adheres to these phases in its research methodology.

\begin{figure}[h]
    \begin{tcolorbox}[arc=0pt,boxrule=0.5pt]
        \centering
        \includegraphics[width=1\linewidth]{chap3/images/dsr_process}
        \caption[DSR Process]{Design Science Research Approach according to Peffers et al.
        Picture:~\cite[p. 72]{sonnenberg2012evaluation}}
        \label{fig:dsr_process}
    \end{tcolorbox}
\end{figure}

\paragraph{``Activity 1 - Problem identification and motivation''~\cite[p. 52]{peffers2007design}:}
``Define the specific research problem and justify the value of a solution.''
~\cite[p. 52]{peffers2007design}

This study aims to improve the prediction of spring back in the air bending process using real world
data with supervised learning.
The value of the solution will be a reduction of trial and error cycles and therefore a reduction of time and cost in
the air bending process.
The problem as well as the motivation for this study was described in the \cref{ch:introduction}
``Introduction'' in more detail.

\paragraph{``Activity 2: Define the objectives for a solution''~\cite[p. 55]{peffers2007design}:}
The goal of a solution comes from the definition of the problem, which takes into account what is possible and feasible.
Objectives can be quantitative or qualitative, and they should be based on the problem specification that was
established in the previous step.
Knowledge about previous solutions and their effectiveness is required~\cite[p. 55]{peffers2007design}.

The objectives for the solution, the different \ac{ML} models, will be to excell in all six Design Principles which
will be introduced in the next chapter.
The Design Principles are a set of criteria that are used to evaluate the performance of the \ac{ML} models.

\paragraph{``Activity 3: Design and development''~\cite[p. 55]{peffers2007design}:}
In this phase, the development of the artifact takes place.
An artifact, which may encompass models, methods, or constructs, serves as a critical element in addressing the
research question.
The process begins with outlining the functionality and architecture of the artifacts, followed by their subsequent
creation~\cite[p. 55]{peffers2007design}.

In this study the artifacts are Machine Learning models.
This steps includes the model selection and the training of the models.
The development of the artifacts is described in the \cref{ch:build} ``Build''.

\paragraph{``Activity 4: Evaluation''~\cite[p. 56]{peffers2007design}:}
In this activity, the effectiveness of the developed artifact in addressing the defined problems from Activity 1 is
examined and assessed.
The evaluator is expected to have knowledge of relevant metrics and analytical techniques.
The evaluation process may vary based on the nature of the problem at hand which is Machine Learning in this
study~\cite[p. 56]{peffers2007design}.
A comparison between the functionality of the artifact and other existing solutions can be conducted.
Additionally, quantifiable parameters can be employed to objectively measure the artifact's
performance~\cite[p. 56]{peffers2007design}.
As already stated in this study the created artifacts are evaluated using six distinct design principles.

\cite{peffers2007design} note that researches are not obligated to follow the order of activities
from 1 to 6
~\cite[p. 56]{peffers2007design}.
In this study the order fo the activities 4 and 5 swapped, in the original DSR approach the evaluation is done after
the demonstration activity, this was changed because the evaluation activity is used to select the best performing
artifacts.
How the \ac{ML} models where evaluated is described in the \cref{sec:evaluation-of-machine-learning-models}.


\paragraph{Activity 5: Demonstration}
The purpose of this activity is to demonstrate the effectiveness of the previously created artifact in solving one or
more problems.
Effective knowledge of the artifact is necessary for this activity
~\cite[p. 55]{peffers2007design}.
Unlike the evaluation activity, which assesses the overall performance of the artifact, demonstration focuses on the
performance of the artifact in specific scenarios.
To achieve this, three scenarios, each comprising two examples (single spring back predictions), are utilized to
compare the best-performing models.
The scenarios are selected to cover the entire attribute space of the problem and are described in detail in
~\cref{ch:evaluation}, \textit{Evaluation}.

\paragraph{Activity 6 - Communication}
The last activity is ued to communicate significance of the problem and the usefulness of the artifacts to a wider
audience
~\cite[p. 56]{peffers2007design}.
In this study this is done in the \cref{sec:results-and-discussion} ``Results and Discussion''.


\section{Design Principles}\label{sec:design-principles}
Recent studies suggest that the desired result of design science endeavors should also be knowledge that takes the
shape of
\ac{DP}~\cite{baskerville2010explanatory, sein2011action, gregor2013positioning}.
They capture knowledge concerning the production of additional examples of objects that belong to the same
category
~\cite[p. 39]{sein2011action}.
Other ML models are examples of the same category in the context of this study.
This information is gathered as the researcher progresses from one instance of the artifact to a more general level
~\cite[p. 37]{chandra2016making}.

In this study \ac{DP}s should provide a set of guidelines for the creation of new artifacts.
% TODO weg weil ich es nicht in der Quelle gefunden habe...
%Also DPs should be abstract enough to be used in later research as well~\cite[p. 37]{sein2011action}.


\section{Evaluation of Machine Learning Models}\label{sec:evaluation-of-machine-learning-models}
The field of software engineering has established standards for determining the quality of software systems and their
components.
One such notable standard is the ISO/IEC 9126, which provides a quality model that is widely used in
software engineering.
These standards are not applicable for evaluating \ac{ML} models, as data-driven software components, such
as \ac{ML} models, have functionalities that is not fully defined by the developer but rather is learnt from data.
As a result, when compared to traditional software engineering, using ML poses unique issues.
~\cite[p. 2]{siebert2022construction}.
Therefore, as noted by~\cite{siebert2022construction}, adaptation is necessary.
They have reinterpreted and expanded upon these existing quality models to make them applicable to the context of
Machine Learning, as stated in their work
~\cite[p. 1]{siebert2022construction}.

In this study, in order to define the Design Principles for the \ac{ML} models in this work, the
considerations of~\cite[]{siebert2022construction} are used.
This allows for a systematic process to assess the quality of the developed models.
\cite{siebert2022construction} consider several quality attributes, including
correctness, relevance, robustness, stability, interpretability and resource utilization.
Alongside these attributes, the authors also provide a set of quality measures to evaluate the
models, but these are not complete and are only used as a starting point for the evaluation of the
models in this work, further evaluation is described in \cref{ch:evaluation}.


\section{Goal Question Metric Approach}\label{sec:goal-question-metric-approach}
To make the defined quality attributes measurable, the Goal-Question-Metric (\ac{GQM})-approach was chosen in this
study.
It is one of the most common approaches in DSR and is divided into three levels~\cite[p. 3]{van2002goal}.

\paragraph{1. Conceptual level (goal):}
``A goal is defined for an object, for a variety of reasons,
with respect to various models of quality, from various points of view, relative to a
particular environment.''
~\cite[p. 3]{van2002goal}

Specific goals are established for a software product (in this case a \ac{ML} model).
These goals consider various quality aspects, perspectives, and contexts.
Clearly defined goals help to guide the direction of the development
~\cite[p. 3]{van2002goal}.
In this study the goals are the Design Principles that are to be established in the next \cref{ch:build}.

\paragraph{2. Operational level (question):}
To achieve the set goals, one or multiple questions are formulated to assess and characterize the objectives.
These questions address the quality attributes and help to understand the object of measurement from the selected
viewpoint.
The questions guide the measurement process by identifying the relevant aspects that need to be
analyzed~\cite[p. 3]{van2002goal}.

\paragraph{3. Quantitative level (metric):}
For each question, a set of metrics is defined to provide quantitative data and answers.
These metrics are used to evaluate the object of measurement, offering precise and objective insights into the
quality aspects being assessed~\cite[p. 3]{van2002goal}.

In this study only one question is formulated for each goal.
The goals, questions and metrics are presented in the next chapter.
