% NOTES 
% Wie ich den sprinback gemssen habe 

\chapter{Build}\label{ch:build}
This chapter outlines the steps of the \ac{DSR} process, which are
summarized under the ``Build'' phase, as described chapter~\ref{ch:research-methodology}
resarch methodology.
The first step in this phase is to identify the problem that need to be addressed,
this step leads to the formulation of \ac{DP}, which are later used to evaluate
the effectiveness of the artefact.
The \ac{DP} are derived from the requirements of the overall models and are based on
the work of~\cite{siebert2022construction}.

The Build phase involves the creation of machine learning models, which serve as
artifacts.
By implementing these steps, the DSR process establishes a rigourous and methodical
approach for creating new artifacts.


\section{Problem Identification and Motivation}\label{sec:problem-identification-and-
motivation}

For several decades, researchers have been conducting extensive research on sheet metal
forming technologies in response to the growing demand for lightweight metal components.
One of the challenges is the phenomenon of spring back, as discuess in~\ref{sec:spring-
back}.
As a result, various efforts have been made to devise methods for compensation~\cite[p.1]{
    liu_newmachinelearning_2012}.

Sheet metal forming is a complex process due to non-linear behavior caused by
deformations of the metal sheet, making the spring back difficult to predict.
Consequently, conventional methods often employ trail and error approaches~
\cite[p. 1]{dib_singleensembleclassifiers_2020}.
Through speaking with experts of the field of sheet metal forming, it was found that
a usual trial and error approach is the creation of so called``technology tables'' that
contain the bending parameters and resulting spring back data.
These tables are produced by conducting numerous experiments with varying bending
angles and metal sheets.
However, this process is time-consuming and costly, making it
unsuitable for the production of high-volume and low-cost components.

\textit{SOURCE: Peter and Dr. Hochstrate but only while talking to them. Find a proper
source}


\section{Objectives of a Solution}\label{sec:objectives-of-a-solution}
The design principles used to evaluate the machine learning (ML) models in \ac{DSR}
process and form the objectives of the solutoin.
work are based on a selection of quality parameters proposed by
Siebert et al.~\cite{siebert2022construction}.
These quality parameters have been tailored specifically for the evaluation of ML
models and are thus well-suited to the evaluation of the models developed in this
thesis.

\subsubsection*{Design Principle 1: Correctness}
\textit{"Does the artifact predict the spring back of a sheet metal with a
high accuracy and
correctness?"}~\cite[p. 16]{siebert2022construction}

The demand for high-quality products is a driving force in the manufacturing industry,
as consumers and businesses alike demand products that meet stringent standard of
precision and accuracy.
This is particularly true in the case of sheet metal forming, where even small
deviations from the desired dimensions can cause significant implications for the final
product.
Therefore, metal parts need to be produced with high precision and accuracy and
spring back is an undesired side effect which needs to be minimized~\cite[p.1]{
    cruz_applicationmachinelearning_2021}.

\ac{ML} has emerged as a powerful tool for predicting the spring back in other bending
methods as shown in section~\ref{sec:state-of-research}~state of research.
A \ac{ML} model should predict the spring back with a high degree of accuracy and
correctness, as even small errors in the prediction will cause fitting problem in the
manufacturing process.

\subsubsection*{Design Principle 2: Relevance}
In addition to measure the correctness it is important to understand ``why''
the learner has this performance.
A model is considered relevant when it is able to accurately predict the outcomes of
new unseen data, that were not used during training.
A relevant model should have a low error rate on both the training data and the test
data and achieve a good bias-variance trade-off~\cite[p. 16]{siebert2022construction}.

Understanding the bias-variance trade-off is crucial in developing \ac{ML} models that
generalize well to unseen data~\cite[p. 49--51]{zhou_machinelearning_2021}.
In machine learning (\ac{ML}), achieving a good bias-variance balance is essential.
This means that a model can accurately capture the underlying patterns in the data
without overfitting or underfitting.

A good bias-variance is essential a \ac{ML} it means that a model can
accurately capture the underlying patterns without overfitting or underfitting to the
data~\cite[p. 49--51]{zhou_machinelearning_2021}.

\subsubsection*{Design Principle 3: Robustness}

When working with real-world data, quality issues such as outliers, missing data, and
noise are common problems. These issues can have a negative impact on the performance
of the model, making it challenging to produce accurate predictions. To overcome these
challenges, a robust model must be able to handle data quality issues and still produce
accurate predictions~\cite[p. 16]{siebert2022construction}.

\cite{saez_evaluatingclassifierbehavior_2016} proposed a new measure called Equalized
Loss of Accuracy to evaluate classifications models for robustness.
Because regressions algorithms are used in this new metrics have to be found~\cite[p.
3]{saez_evaluatingclassifierbehavior_2016}.

\subsubsection*{Design Principle 4: Stability}

The stability of a model is an important quality parameter, as it is essential that
the model produces the same results when trained on different datasets.
This is particularly important in the case of \ac{ML} models, as the training data
is often limited and the model is trained on a small subset of the available data.
Therefore, it is important that the model is able to generalize well to unseen data and
produce repeatable results when trained on different datasets~\cite[p. 16]{
    siebert2022construction}.

\subsubsection*{Design Principle 5: Interpretability}

Interpretability refers to the ease with which humans can understand and make
sense of the decisions made by a trained machine learning model~\cite[p. 13]{
    molnar2020interpretable}.
\cite{miller2019explanation} define interpretability as the ``degree to which
a human can understand the cause of a decisio''~\cite[p. 1]{miller2019explanation}.
Good interpretability is important because it allows users to trust and rely
on the model, and it can also help with debugging and improving the model (\textit{
    Source}).

Interpretable models will also deliver more insights for this project, as the
goal is to understand the relationship between the features and the target outcome.

As mentioned in chapter~\ref{ch:theoretical-foundations} there are many parameters
and variables involved in the sheet metal forming process.
That makes the process design quite complex, particularly in the production
of components which require several stages, and thus more than one set of tools~\cite[p.
1]{dib_singleensembleclassifiers_2020}.
A interpretable model which allows conclusions how the results where generated is better.

\subsection*{Design Principle 6: Resource utilization}
One of the main objectives of using machine learning to predict spring back is to
minimize the number of trial-and-error cycles in the manufacturing process and save
resources (see Section \ref{sec:problem-identification-and-motivation}). However, it is
important to consider that creating an ML model also requires resources.

Therefore, it
is crucial to take into account the resources needed to train and make predictions with
the model~\cite[p. 16]{siebert2022construction}.


\section{Dataset generation}\label{sec:dataset-generation}

To generate the dataset, bending experiments on metal sheets of varying
thicknesses where conducted.
Specifically, cold rolled steel sheets that conform to the DIN EN
10130 with thicknesses raning from 0.5 mm to 3 mm where used.
The material was selected because it is commonly used in bending processes and widely
available.
To create the dataset, pieces of the material were cut into 20×100 mm pieced each piece
was bent using a Zwick three-point-bending machine.
It was paid attention to ensure that each metal sheet was bent in the same
rolling direction as preliminary tests showed that the spring back differed depending
on the rolling direction.

The dataset consisted of 384 \texttt{.tra} files which is a proprietary format used by
the machine.
Python script where developed to covert the output data format from the
machine to CSV files.

The following describes the experimental setup used for the experiments
performed.
The data collection was done outside of the thesis period and originally consited of 384
samples. \texttt{X} samples where added later to the dataset to fill gaps present in
the dataset.

% Pado: Since the data collection was technically done outside of the thesis
% period, 4.2.1
% (Multiple cycles) can be
% omitted. 4.2.2 is needed in order to understand the data set.

%\subsection{Preliminary Tests}
%A number of preliminary tests were conducted to determine the influence of
%the punch penetration
%on the spring back.
%
%\subsubsection{Multiple Cycles}
%One approach was to test if multiple spring back can be measure using only
%one sheet.
%Therefore, the machine was programmed to perform multiple cycles in one
%attempt and bend the
%metal sheet multiple
%times. The benefit of this approach would have been a faster generation of
%the dataset because
%spring backs could be
%measured in on attempt, also less material would have been used.
%
%Figure~\ref{springback_multiple} shows one of these attempts. The metal
%sheet was bent 4 times
%using $y_p$ values
%from 5 to 8. The results show, that 4 different spring backs can be
%measured, but the spring back
%does not vary like
%expected. It was observed as well, that the spring backs are different in
%every attempt, this is
%shown in
%Figure~\ref{springback_multiple_inconsistent_results}.
%Bending 4 different metal sheets each only one time returned very different
%results.
%A possible explanation could be the cold deformation of the steel, which is
%not reversible.
%Because this approach did
%not work, the machine was programmed to perform one cycle at a time.
%
%\captionsetup{width=0.45\textwidth}
%
%\begin{figure}[H]
%    \centering
%    \begin{minipage}[b]{0.5\textwidth}
%        \centering
%        \includegraphics[width=0.9\textwidth]{springback_multiple.jpg} %
%first figure itself
%        \caption{Experiment: Bending one metal sheet multiple times with
%different $y_p$ values.}
%        \label{springback_multiple}
%    \end{minipage}\hfill
%    \begin{minipage}[b]{0.5\textwidth}
%        \centering
%        \includegraphics[width=1
%.1\textwidth]{springback_multiple_inconsistent_results.png} %
%        second figure itself
%        \caption{Inconsisten results bending one metal sheet mutliple times.
%The spread of the
%        results is very large.}
%        \label{springback_multiple_inconsistent_results}
%    \end{minipage}
%    \label{fig:springback_multiple_overview}
%\end{figure}

%\subsubsection{Bending Machine}
%Before using the three point bending machine, a brake bending machine was
%used to test the
%influence of the bending
%on the spring back. The brake bending machine is a machine used to bend
%metal sheets. It is a
%very common machine in
%the industry and is used to bend metal sheets to a specific radius. The
%brake bending machine
%used is a
%\textit{Bendmaster 1000} from \textit{Bendmaster}.
%
%After a series of bends it was observed, that the spring back values where
%much higher than
%expected. The explanation
%for that behavior was, that altering the position the bending beam of that
%specific machine was
%not enough to get the
%desired angle. Thus, the machine excluded for the generation of the data and
%the three point
%bending machine was used
%instead.
%
%Despite the inaccurate data, it was later observed, that the distribution of
%the spring backs was
%very similar to the
%later experiments with the three point bending machine.

\subsection{Experimental setup}\label{subsec:experimental-setup}
The experimental setup comprises of a three-point bending machine, consisting
of a punch and die, with the latter lacking a bottom, which allows only air bending.

The material testing machine utilized is the \textit{Zwick MX 25A}, which is
equipped with a load cell and a displacement sensor.
The load cell measures the force applied to the sheet measured in $N$), while the
displacement sensor measures the displacement of the punch, denoted as $y_p$.
The punch is mounted on the top of the machine and is stationary, while the
die is mounted on the bottom and is the part which can be moved.
The die opening of the machine is adjustable with from 10mm to 100 mm.
The machine is operated via a computer and the \textit{ZwickRöll TestXpert} software,
which is used for both machine control and data collection.

The experimental setup and the process parameters are shown in
Figure~\ref{fig:process_parameters} where $V$ is the die opening which is the
opening of the two contact points on which the sheet metal is placed.
The parameter $y_p$ is the punch penetration, which is the distance the punch is moved
into the sheet.
The metal sheet thickness is donated as $t$ and the bending angle as $\alpha$.
The parameter $r_p$ is the radius which of the tip of the punch, which was never
replaced in the experimental setup and therefore remains constant.

\begin{figure}[h]
    \begin{tcolorbox}[arc=0pt,boxrule=0.5pt]
        \centering
        \includegraphics[trim=left botm right top, width=0.6\textwidth,
            clip]{process_parameters}
        \caption{\textbf{Process parameters:} Sheet bending angle ($\alpha$), sheet
        thickness ($t$), punch
        penetration ($y_p$), die opening ($V$) and punch radius ($r_p$)}
        \label{fig:process_parameters}
    \end{tcolorbox}
\end{figure}


To ensure consistent results, a set of constant and variable parameters were selected.
The constant parameters consisted of the punch-and-die tooling made of steel, which
included the die punch radius (r_p), as well as the length and width of the
metal sheet. Additionally, the punch travel speed, hold time, and punch force threshold
were also considered constant.

All metal sheets used in the setup were standardized with a length of 80 mm and a width
of 20 mm.
The hold time, which refers to the duration that the punch remains stationary
after reaching the maximum displacement ($y_p_{\max}$), was set to a minimum of 1 second.
The punch force threshold was set to 1 N, meaning that the punch was initially moved at
a higher speed until the force reached 1 N, and then moved at a slower speed of 80 mm
/min until the $y_p_{\max}$ was reached.


\begin{table}[htb]
    \begin{tcolorbox}[arc=0pt,boxrule=0.5pt]
        \sisetup{group-minimum-digits = 4}
        \centering
        \label{tab:experimental-setup-constant-parameters}
        \begin{tabular}{lll}
            \toprule
            \thead{\textbf{Parameter}} & \thead{\textbf{Values}} &
            \thead{\textbf{Unit}}
            \\
            \midrule
            Punch radius & 5 & $mm$
            \\
            \hdashline
            Sheet width & 20 & $mm$
            \\
            \hdashline
            Sheet length & 100 & $mm$
            \\
            \hdashline
            Punch speed & 80 &
            $mm/min$ \\
            \hdashline
            Punch speed up (after bend) & 8 &
            $mm/min$ \\
            \hdashline
            Hold time & 1 & $s$ \\
            \hdashline
            Punch force threshold & 1 & $N$
            \\
            \bottomrule
        \end{tabular}
        \caption{Constant parameters in th eperimental setup}
    \end{tcolorbox}
\end{table}

The experiment involved varying three parameters: the die opening ($V$), the maximum
punch penetration ($y_p$), and the thickness of the metal sheet ($t$). The die opening
was varied from 10 mm to 50 mm, while the maximum punch penetration was varied from 2
.5 mm to 20 mm. The thickness of the metal sheet was also varied, with values ranging
from 0.5 mm to 2 mm.
These parameters and their values can be seen in Table~\ref{tab:experimental-setup-
variable-parameters}.

\begin{table}[htb]
    \begin{tcolorbox}[arc=0pt,boxrule=0.5pt]
        \sisetup{group-minimum-digits = 4}
        \centering
        \label{tab:experimental-setup-variable-parameters}
        \begin{tabular}{lll}
            \toprule
            \thead{\textbf{Parameter}} & \thead{\textbf{Values}} &
            \thead{\textbf{Unit}}
            \\
            \midrule
            %  \unit{(Kcal\per\mole)\squared}}} & \thead{RMSD l.b.} &
            %  \thead{RMSD u.b.}  \\
            \midrule
            Punch penetration  $y_p$ & 2.5, 5, 7.5, 10, 12.5, 15, 17.5, 20 &
            mm \\
            \hdashline
            Die opening        $V$ & 10, 20, 30, 40, 50
            & mm \\
            \hdashline
            Thickness          $t$ & 0,5, 1, 1.5, 2, 2.5, 3
            & mm \\
            \bottomrule
        \end{tabular}
        \caption{Varying parameters in the experimental setup}
    \end{tcolorbox}
\end{table}

\subsection{Measuring The Spring Back} \label{subsec:measuring_the_spring_back}
The output data included various data points that were used to calculate the spring back.
Key parameters for this calculation were the force (``Standardkraft''), punch penetration
(``Standardweg''), and time (``Testzeit'').
Figure~\ref{fig:springback_measured}
illustrates these three parameters, with the blue line representing the force and the
gray line representing the punch travel.


In Figure~\ref{fig:springback_measured} it can be seen that the force jumps to around 10 N
as soon as the punch touches the metal.
This is the case because the travel speed of the punch is 80 mm/min which relatively fast.
It was set to that speed in order to reduce the time of the experiments.

The wait time at of 1 second $y_p_{max}$ is a limitation of the machine and can not be
changed and therefore is always a part of the experimental setup.

\textit{Explain the dip}

For a short time after the lift, the load cell still measures a force.
That is because the metal sheet springs back and the punch is still in contact with the
sheet.
This was measured using a python script,the green and the yellow point represent the
resulting spring back distance.

\begin{figure}[H]
    \begin{tcolorbox}[arc=0pt,boxrule=0.5pt]
        \centering
        \includegraphics[width=0.9\textwidth]{springback_measured}
    \end{tcolorbox}
    \caption{A steel metal sheet was bent with a punch penetration of 5
    mm the spring back is 0
    .37 mm. The blue line
    shows the force and the blue line shows the punch penetration.}
    \label{fig:springback_measured}
\end{figure}

% Pado: This is a little confusing - does the Y axis show the force? Why is
% the blue line the
% shape it is? Is it
% maybe two separate lines, the movement of the punch and the reaction of the
% metal?

\subsection{Dataset Exploration}\label{subsec:dataset-exploration}
The dataset was explored using the python library \textit{pandas}~\cite{mckinney-proc-
scipy-2010} and the
\textit{matplotlib}~\cite{Hunter:2007} library.

The goal of this section is to give the reader an overview of the dataset and
to show the relationship between the features and the dependent variable.

\subsubsection{Features}
The output data of the bending machine consisted of 26 features, which are listed in
the appendix.
However, only three features - force, distance $y_p$, and time- are relevant for
calculating the spring back, as described in the previous section (
Section~\ref{subsec:measuring_the_spring_back}).
These three features were combined with the calculated spring back to form the final
dataset, which contained a total of
396 data points generated using the method described earlier.
An example of the dataset is presented in Table~\ref{tab:dataset_example}.

\begin{table}[H]
    \begin{tcolorbox}[arc=0pt,boxrule=0.5pt]
        \sisetup{group-minimum-digits = 4}
        \centering
        \begin{tabular}{l|llll}
            \toprule
            \thead{\textbf{index}} & \thead{\textbf{Distance}} &
            \thead{\textbf{Spring Back}} &
            \thead{\textbf{Thickness}}
            & \thead{\textbf{Die Opening}}
            \\
            1   & 5.0  & 0.6667 & 2.0 & 50  \\
            \hdashline
            2   & 15.0 & 0.9164 & 2.0 & 50  \\
            \hdashline
            3   & 10.0 & 0.6829 & 2.0 & 50  \\
            \hdashline
            ... & ...  & ...    & ... & ... \\
            \hdashline
            396 & 5.0  & 0.6667 & 3.0 & 10  \\
            \bottomrule
        \end{tabular}
    \end{tcolorbox}
    \caption{Varying parameters in the experimental setup}
    \label{tab:dataset_example}
\end{table}

Figure~\ref{fig:v30_springbacks} illustrates the spring backs observed in the $V30$
dataset.
Two general trends are evident.
First, a lower thickness of the metal sheet results in a higher spring back.
Second, a higher punch penetration results in a higher spring back.
Also it non-linear relationship between the punch penetration and the spring back can
be observed.

In the chosen example it chan be observed that the spring back of the metal sheets with $t = 0.5$
are significantly higher then the other thicknesses.
Despite the availability of data, the factors underlying the spring back behavior
remain incompletely understood.
Because too many factors are involved in the bending process.

\begin{figure}[htb]
    \begin{tcolorbox}[arc=0pt,boxrule=0.5pt]
        \centering
        \includegraphics[width=0.7\textwidth]{all-springbacks-consolidated.png}
        \caption{Springbacks for $V30$}
        \label{fig:v30_springbacks}
    \end{tcolorbox}
\end{figure}

\subsubsection{Data Quality}
The dataset was created in a controlled experimental environment, where
was given to ensure that the samples were measured with precision and accuracy.
This results in a dataset that contains a minimal number of outliers and a
high level of data quality, which is an essential requirement for reliable machine
learning models.

As depicted in Figure~\ref{fig:train_test_split}, the dataset covers all possible
combinations of the process parameters, $V$, $t$ and $y_p$.
The $y_p$ values in the dataset are uniformly distributed and always range between 2.5
and 20 mm.
The dataset was continuously expanded with new data points throughout the project to
increase its size and diversity.

During the data collection and expansion process, several outliers and incorrect
measurements were identified and removed from the dataset to ensure its high quality.
Although this dataset has been developed in a controlled environment, it may not
accurately represent real-world scenarios, where the data quality is often affected by
a range of factors such as measurement errors, noise, and bias.
This will be taken into account in later sections, where data quality issues will be
added in order to evaluate the robustness of the models.

\subsection{Data Preprocessing}\label{subsec:data-preprocessing}
The three independent features $y_p$, $V$ and $t$ as well as the dependent
feature $spring\_back$ were normalized using the \texttt{MinMaxScaler} or in some cases
the \texttt{StandardScaler} from the \texttt{scikit-learn}\cite{scikit-learn}.
The\texttt{MinMaxScaler} scales the data between 0 and 1.
The scaler was fitted on the training data and then used to transform the test data.
The scaler was saved to be used for the prediction of the spring back of the real world
data.

Scaling is only done on the training data, because cross-validation is later used to
tune and evaluate the models.
Scaling the whole data set before the split would lead to data leakage because the
minimum and maximum values of the test data would be used to scale the training data.
How the data was split can be seen in Figure~\ref{fig:train_test_split}.

\subsection{Computational Setup}\label{subsec:computational-setup}
For training the machine learning models a ThinkPad X1 Carbon 2019 with an
Intel Core i7-10610U CPU @ 1.80GHz and 16 GB RAM was used.
The operating system used is Ubuntu 20.04.2 LTS. The code for the model is
written in Python 3.8.5 using the IDE PyCharm.
The libraries used are mentioned in Table~\ref{table:libraries}.

\captionsetup{width=1\textwidth}

\begin{table}[htb]
    \begin{tcolorbox}[arc=0pt,boxrule=0.5pt]
        \sisetup{group-minimum-digits = 4}
        \centering
        \begin{tabular}{ll}
            \toprule
            \thead{\textbf{Library}} & \thead{\textbf{Version}} &
            \toprule
            numpy~\cite{harris2020array} & 1.23.2 \\
            \hdashline
            pandas~\cite{mckinney-proc-scipy-2010} & 1.5.1 \\
            \hdashline
            matplotlib~\cite{Hunter:2007} & 3.6.2 \\ \hline
            \hdashline
            scienceplots~\cite{SciencePlots} & 2.0.1 \\
            \bottomrule
        \end{tabular}
        \caption{Libraries used for the machine learning models.}
        \label{table:libraries}
    \end{tcolorbox}
\end{table}

Upon examining the correlation matrix depicted in Figure~\ref{fig:correlation_matrix},
it is evident that the distance and spring back features exhibit a stronger correlation
than the other features.
This correlation is to be expected since punch penetration $y_p$ is the primary factor
that influences the amount of spring back.
It is noteworthy that the other features are not correlated with each other, indicating
the absence of multicollinearity, which is a desirable trait for machine learning models.

\begin{figure}[H]
    \begin{tcolorbox}[arc=0pt,boxrule=0.5pt]
        \centering
        \includegraphics[width=0.7\textwidth]{correlation_matrix}
        \caption{Correlation matrix}
        \label{fig:correlation_matrix}
    \end{tcolorbox}
\end{figure}


\section{Model Selection}\label{sec:model-selection}

\subsection{Support Vector Regression (SVR)}\label{subsec:support-vector-regression-(svr)}
Support Vector Machines \ac{SVM} are usually used for classification problems.
The \ac{SVM}algorithm is used to find a hyperplane in an N-dimensional space (N - the
number of features) that distinctly classifies the data points~\cite[p. 42]{
    awad_efficientlearningmachines_2015}.
Predicting the spring back is a regression problem, so the \ac{SVM} algorithm
need to be
generalized of
classification problems, where the model returns continuous values instead of
finite set of values.
This is done by using the \ac{SVR} algorithm, which is inspired by  \ac{SVM}
and uses the same
principles.
It fits a model to data using only residuals smaller in absolute value than a
certain constant
called
$\epsilon$-sensitivity

This is done by creating a "tube" of $\epsilon$ width around the data, with
points inside the
tube not being
penalized and points outside the tube being penalized based on their distance
from the predicted
function.
This is done similar to how \ac{SVM}s penalize points in classification.
Like \ac{SVM}, \ac{SVR} fins a well-fitting hyperplane to a kernel-induced
feature space to
achieve good
generalization performance using the original features.
\cite[p. 369]{montesinoslopez_supportvectormachines_2022}

\subsubsection*{Kernel Trick}

The kernel trick is a method to transform the data into a higher dimensional
space, where the
data is linearly
separable.
This is done by using a kernel function, which is a function that maps the
data into a higher
dimensional space.
Two methods are usually used for \ac{SVM}s, the polynomial kernel and the
radial basis function,
also known as gaussian kernel.
\cite[p. 97-98]{muller_introductionmachinelearning_2016}
In practise the mathematical details behind these kernels are not important,
but it is important
to know that they are used to transform the data into a higher dimensional
space, where the data
is linearly separable.

\subsection{Multi Linear Regression}\label{subsec:multi-linear-regression}

\subsection{Polynomial Regression}\label{subsec:polynomial-regression}

\subsection{Decision Trees}\label{subsec:decision-tree-regression}
\ac{DT}s are a commonly utilized type of model for categorizing and
predicting outcomes.
They construct hierarchy if rules to solve classification or regression
problems~\cite[p.
70]{muller_introductionmachinelearning_2016}.

In \ac{DT} models, the data is divided into multiple segments based on
specific feature values.
This division creates various subsets of the dataset, with each sample being
a part of one subset.
The last subsets are referred to as terminal or leave nodes while the
intermediary subsets are
known as internal nodes or split nodes.
To predict the outcome in a particular leaf node, the mean outcome of the
training data present
in that nodes is taken into account~\cite[p. 76]{molnar2020interpretable}.

\subsection{Random Forest Regression}\label{subsec:random-forest-regression}
% Erst einmal auf decision tree generell eingehen
A commonly used method in machine learning. The goal is to solve
classification or regression
problems by predicting
the value of a output variable by one or multiple input variables. \cite[p.
253]{shaik_briefsurveyrandom_2019}
To build a \ac{DT} the source dataset represents the root node of the tree
this data set is split
into leafs
(children) by using a set of spitting rules until each leaf in the \ac{DT} is
"pure" and only
contains one target
value. Depending on the use cases this is a single class or a single
regression value. \cite[p.
70-72]{muller_introductionmachinelearning_2016}
% -> Problem von decision tree 
The main drawbback of \ac{DT}s is the tendency to overfit and poor
generalization performance,
what makes them not
paticaly for most use cases. Therefore usualy ensemble methods are used
instead of a single
\ac{DT}. \cite[p.
78]{muller_introductionmachinelearning_2016} \cite[p.
251]{liu_newmachinelearning_2012}
% -> Warum random forst das problem löst 
Random forest \cite[]{breiman_randomforests_2001} is a type of ensemble
learning algorithm in
which multiple decision
trees, which are "weak learners," are trained and combined to produce a more
accurate and stable
prediction, known as
a "strong learner." \cite[p. 24]{awad_efficientlearningmachines_2015}
The risk of overfitting is mitigated by subset and feature randomization.
Each root node uses a
unique subset of the
data and each leaf is split using a random features. This ensures that no
single tree sees all of
the data, allowing
the model to focus on general patterns rather than being sensitive to noise.
\cite[p.
251]{liu_newmachinelearning_2012}
% -> Beschreibung Random forest / Random foret regression 
In this supervised learning method,
%which was influenced by the research of Amit and Geman (1997), Ho (1998),
%and Dietterich (2000),
a "divide and conquer" approach is used. This involves dividing the data into
smaller samples,
incrementally building
a randomized tree predictor for each sample, and then combining (aggregating)
these predictors
together. This
approach has proven to be effective. Because not only one but multiple
classifiers are used the
random forest
learning is known as ensemble model. \cite[p. 254]{shaik_briefsurveyrandom_2019}

% -> Vorteile / Nachteile Random Forest 
This mechanism is flexible enough to handle classifications and regression
problems, this is one
of the reasons that
random forests count to the most successful \ac{ML} methods. \cite[p.
3-4]{biau_randomforestguided_2016} \cite[p.
25]{breiman_randomforests_2001}

%
% Advantages
%
Random forests are a type of machine learning algorithm that uses bagging and
the random
selection of features to
produce accurate results. They are effective at handling noise and can work
with both continuous
and categorical
variables. This combination of techniques helps improve the performance of
the algorithm. \cite[p.
259]{liu_newmachinelearning_2012}
Decision trees have a limitation in their ability to overfit, which is a
disadvantage. This is
mitigated by the use
of subset and feature randomization. Specifically, each base model uses a
unique subset of the
data, and each node in
the decision tree is split using a random set of features. This ensures that
no single tree sees
all of the data,
allowing the model to focus on general patterns rather than being sensitive
to noise. \cite[p.
259]{liu_newmachinelearning_2012}

% Bagging technique helps the algorithms performance. 
% Works well out of the box with no hyperparameter tuning.
% Fast, robust, and can show feature importances which can be quite useful.

% Disadvantages
%

% Boosted algos usuallaly outperform Random Forest.
% RF is not able to extrapolate based on the data. It can only predict within
% the range of the
% training data.
% Random forest regressor is unalbe to discover trens that would enable it in
% extrapolating
% values that fall outside
% of the training set.

%  Extrapolating is the process of estimating or predicting something beyond
%  the range of
%  available data. (chatgpt)

%
% Alogithm steps 
%

% The following Random Forest Algorithm [9] gives the steps in constructing the
% decision trees.
% • Take N as the number of training data instances in the samples. Let M be the
% number of attributes in given input dataset.
% • Let m be the Number of parameter in the input that determines the next
% attribute
% to be chosen at each tree node; (where mislesserthan M).
% • The training samples are taken and a tree is constructed for each sample
% with
% replacement.
% • For tree node, arbitrarily select m attributes in that particular node.
% • The best split is computed based on the m input attributes of the sample
% dataset.
% • Each tree is grown without pruning.
% \cite[p. 254-255]{shaik_briefsurveyrandom_2019}

% A random forest does not require any cross verification and it is not
% over-fitting [8]. The
% Random forest uses Adaboost and Bootstrapping techniques to construct multiple
% classifiers. \cite[p. 254]{shaik_briefsurveyrandom_2019} 


% In a classification random forest, each tree in the forest makes a
% prediction for the class
% label of a given
% example, and the final prediction is made by majority vote, with the class
% label that receives
% the most votes being
% chosen as the prediction. In a regression random forest, each tree in the
% forest makes a
% prediction for the
% continuous value, and the final prediction is made by averaging the
% predictions of all the
% trees in the forest.
% (chatgpt)

% In both cases, the random forest model creates a large number of decision
% trees and trains them
% on different
% subsets of the training data. The decision trees are trained using a
% technique called bagging,
% which involves
% randomly selecting a subset of the training examples for each tree and
% training the tree on
% that subset. The idea
% behind bagging is to train each tree on a slightly different subset of the
% data, which can help
% to reduce
% overfitting and improve the overall performance of the model. (chat gpt)

\subsubsection{Gradient Boosting Regression Tree}
A gradient boosting regression is a type of ensemble learning 2algorithm in
which multiple
decision trees are combines
to produce a more accurate and stable prediction. Similar to the random
forest algorithm gradient
boosting combines
multiple weak learners to create a strong learner.
The difference to a random forest is, that the trees are trained in a serial
manner and each tree
corrects the errors
of the previous tree. \cite[p. 88-89]{muller_introductionmachinelearning_2016}
Gradient boosted tree use strong pre-pruning and therefore produce shallow
trees with a depth of
one to five. This
brings the advantage of a smaller model which uses less memory and also
results in a faster
prediction.
Usually generating more trees improves the overall performance of the model.
\cite[p.
88-89]{muller_introductionmachinelearning_2016}
Also the algorithm performs well without scaling the dataset and can handle a
mixture of binary
and continuous
features. \cite[p. 88-89]{muller_introductionmachinelearning_2016}
Like other tree-based models it does not perform well on high-dimensional data.

\subsubsection{Multi-layer Perceptron}

\textit{Should I a MLP or will it be out of scope?}

\subsubsection*{Results}


\section{Model Training}\label{sec:model-training}

\subsection{Training-Test Split}\label{subsec:training-test-split}
Figure~\ref{fig:train_test_split} shows the data set and which parts of it is
used for training
and testing the used
models. Samples with a die opening of 30 are used to test the performance and
the remaining part
is used for training.
A different approach would be to us a random test and train split, this would
lead to a better
performance of the
models but would not evaluate their ability to predict new data of a
different die opening.
The die opening 30 was chosen because it is in the middle of the selected
data set and therefore
the models should be
able to predict the data of this die opening.
It is expected that this approach will lead to improved model performance as
the data removed
lies in the middle of the parameter space.
This should result in the model generalizing better on that specific range of
data as it has not
been over-exposed to it during the training process.

All models are trained with the same data set and the same parameters. The
only difference is the
used algorithm.

\begin{figure}[H]
    \begin{tcolorbox}[arc=0pt,boxrule=0.5pt]
        \centering
        \includegraphics[width=0.8\textwidth]{test_train_split}
        \caption{The training and test dataset. (Data especially for V40 is
        still missing)}
        \label{fig:train_test_split}
    \end{tcolorbox}
\end{figure}

Additional to the data split shown in Figure~\ref{fig:train_test_split} a
random test split was
used to train the
models. The random 80/20 split was also used to evaluate the performance of
the models and to
compare them to each
other.

The reason for choosing to different methods for the data split is that the
first one is used to
evaluate the models'
ability to generalize on new data. The second one is used to evaluate the
models' performance on
the data set.
Also in real-world applications is possible that there is already data for
all needed
$V-t$-combinations and models
could be trained on that data.

It is expected that the models perform better on the random data but the
information gain about
the models' ability
to generalize on new data is less because the random split will most
certainly contain data of
all $V-t$-combinations.

\subsubsection*{\textit{Notes}}
\begin{itemize}
    \item \textit{\textbf{Dataset not yet complete}, data for all missing Vt
    combinations will be
    added}
    \item \textit{V10 does differ very much and it might be not good to
    include in the data set.
    Maybe add V60 instead}
\end{itemize}

\subsection{Linear Regression}\label{subsec:linear-regression}
A liner regression model uses the feature inputs to make prediction about the
target by
creating a linear relationship which is easy to understand and
interpret~\cite[p.
37]{molnar2020interpretable}.

Linear models can be used to understand the relationship between a target
variable y and one or
more input features x.
The general formula for predicting in a linear model in regression is shown in
\ref{eq:linear-regression}~\cite[p.
45]{muller_introductionmachinelearning_2016}.
In this context, $x[0]$ to $x[p]$ represent the attributes (with p being the
number of
attributes) of
a specific data point. The parameters $w$ and $b$, which are learned by the
model, and the predicted
output $\hat{y}$, are also included~\cite[p.
45]{muller_introductionmachinelearning_2016}.

\begin{tcolorbox}[arc=0pt,boxrule=0.5pt]
    \begin{equation}
        \hat{y} = w[0] * x[0] + w[1] * x[1] + ... + w[p] * x[p] + b
    \end{equation}
    \label{eq:linear-regression}
\end{tcolorbox}

\ac{LR} models estimate the target value as linear combinations of the
feature inputs.
The linearity of the relationship between the inputs and the target makes the
interpretation of the model straightforward~\cite[p.
37]{molnar2020interpretable}.


Linear regression has no hyperparameters to tune and therefore no
hyperparameter tuning was
performed.

\subsection{Decision Trees}\label{subsec:decision-trees}
Linear and logistic regression models are not effective when the relationship
between input
features and output is not linear or when features have interactions with one
another.

Decision trees are models are useful when the relationship between features
and outcome is
or when features interact with one another.
These models split the data multiple times based on certrain cutoff values,
resulting in
different subset of the dataset.
The final subsets, called leaf nodes are used to predict the outocme by
taking the average of the
training data in that subset~\cite[p. 76]{molnar2020interpretable}.

\subsection{Decision Tree}\label{subsec:decision-tree}
Grid search cross validation was used to find the best hyperparameters for
the decision tree.
All hyperparameters are summarized in Table~\ref{tab:hyperparameters_dt}.

\begin{table}[H]
    \begin{tcolorbox}[arc=0pt,boxrule=0.5pt]
        \sisetup{group-minimum-digits = 4}
        \centering
        \caption{Hyperparameters of the random forest model.}
        \label{tab:hyperparameters_decision_tree}
        \begin{tabular}{llp{7cm}}
            \toprule
            \thead{\textbf{Hyperparameter}} & \thead{\textbf{Value}} &
            \thead{\textbf{Description}}
            \\
            %  \unit{(Kcal\per\mole)\squared}}} & \thead{RMSD l.b.} &
            %  \thead{RMSD u.b.}  \\
            \toprule
            criterion & absolute\_error & The function to measure the quality
            of a
            split. \\
            \hdashline
            max\_depth & 30 & The maximum depth of the tree
            . \\
            \hdashline
            min\_samples\_split & 4 & The minimum number of samples required to
            split an internal node. \\
            \hdashline
            min\_weight\_fraction\_leaf & 0.0 & The minimum weighted fraction
            of the
            sum total of weights (of all the input samples) required to be at
            a leaf node. \\
            \hdashline
            splitter & random & The strategy used to choose
            the split at each node. \\
            \hdashline
            ccp_alpha & 0.0 & Complexity parameter used for Minimal Cost
            \bottomrule
        \end{tabular}
    \end{tcolorbox}
\end{table}

\subsection{Random Forest}\label{subsec:random-forest}

\subsubsection*{Hyperparameter Tuning}
Grid search cross validation was used to find the best hyperparameters for
the random forest
model using
Scikit-Learn's default \texttt{GridSearchCV} function.
All hyperparameters are summarized in Table~\ref{tab:hyperparameters_rf}.

The \textit{criterion} hyperparameter was set to \textit{absolute\_error}
because the absolute
error is the metric
used to evaluate the model.

The \textit{n\_estimators} where set to 10 using grid search cross
validation, because the model
should not be too
complex and the number of trees should not be too high.

The \textit{max\_depth} was set to 10 using grid search cross validation. The
default unpruned
model did overfit the
training data and was not able to generalize on the new data well. This is
expected behavior of
decision trees which
is descriped by \cite[p. 133-136]{muller_introductionmachinelearning_2016}
amongst others.

The \textit{min\_samples\_split} was set to 2 using grid search cross
validation. The default
value of 2 was chosen
because it is the default value of the random forest model in Scikit-Learn.

The \textit{min\_samples\_leaf} was set to 1 using grid search cross
validation. The default
value of 1 was chosen
because it is the default value of the random forest model in Scikit-Learn.

The \textit{max\_features} was set to \textit{auto} and therefor the models
does use all avialble
featues. As
described in Figure~\ref{fig:rf_feature_importance} only limited featues are
avaialbe and all of
them are important
for the depenend variable spring back

\begin{table}[H]
    \begin{tcolorbox}[arc=0pt,boxrule=0.5pt]
        \sisetup{group-minimum-digits = 4}
        \centering
        \caption{Hyperparameters of the random forest model.}
        \label{tab:hyperparameters_rf}
        \begin{tabular}{llp{7cm}}
            \toprule
            \thead{\textbf{Hyperparameter}} & \thead{\textbf{Value}} &
            \thead{\textbf{Description}}
            \\
            %  \unit{(Kcal\per\mole)\squared}}} & \thead{RMSD l.b.} &
            %  \thead{RMSD u.b.}  \\
            \toprule
            n\_estimators & 10 & The number of trees in the forest.
            \\
            \hdashline
            criterion & absolute\_error & The function to measure the quality
            of a
            split. \\
            \hdashline
            max\_depth & 30 & The maximum depth of the tree.
            \\
            \hdashline
            min\_samples\_split & 4 & The minimum number of samples required to
            split an internal node. \\
            \hdashline
            min\_samples\_leaf & 2 & The minimum number of samples required
            to be
            at a leaf node. \\
            \hdashline
            max\_features & auto & The number of features to consider when
            looking for the best split. \\
            \hdashline
            max\_leaf\_nodes & X & Grow trees with max\_leaf\_nodes in
            best-first fashion. \\
            \hdashline
            max\_leaf\_nodes & X & Grow trees with max\_leaf\_nodes in
            best-first fashion. \\
            \bottomrule
        \end{tabular}
    \end{tcolorbox}
\end{table}

\subsection{Gradient Boosted Regression
Trees}\label{subsec:gradient-boosted-regression-trees}

\subsubsection*{Hyperparameter Tuning}

"Gradient boosted trees are frequently the winning entries in machine
learning competitions, and are widely used in industry. They are
generally a bit more sensitive to parameter settings than random
forests, but can provide better accuracy if the parameters are set
correctly." \cite[p. 88-89]{muller_introductionmachinelearning_2016}

"Apart from the pre-pruning and the number of trees in the ensemble,
another important parameter of gradient boosting is the learning rate,
which controls how strongly each tree tries to correct the mistakes of
the previous trees. A higher learning rate means each tree can make
stronger corrections, allowing for more complex models. Adding more trees to
the ensemble, which
can be accomplished
by increasing
n estimators, also increases the model complexity, as the model has
more chances to correct mistakes on the training set." \cite[p.
88-89]{muller_introductionmachinelearning_2016}

"The main parameters of gradient boosted tree models are the number
of trees, n estimators, and the learning rate, which controls the degree to
which each tree is
allowed to correct the
mistakes of the previous trees.
These two parameters are highly interconnected, as a lower
learning rate means that more trees are needed to build a model of
similar complexity. In contrast to random forests, where a higher
n estimators value is always better, increasing n estimators in gradient
boosting leads to a more complex model, which may lead to overfitting. A
common practice is to fit n estimators depending on the time and
memory budget, and then search over different learning rates." \cite[p.
88-89]{muller_introductionmachinelearning_2016}

\subsection{Support Vector Machine}\label{subsec:support-vector-machine}

\subsubsection*{Hyperparameter Tuning}
The \textit{kernel}, \textit{degree}, \texit{gamma} and \textit{epsilon}
where set using grid
search cross validation.
Gamma controls the width of the gaussian kernel, it determines when points
are close or far away.
The C parameter controls the importance of each point.

The features of the dataset have different order of magnitude, this is
already a problem for other
models, but big effects on the kernel \ac{SVM}.
To resolve this problem the data was scaled using the \textit{MinMaxScaler}
between 0 and 1.
The model trained on the scaled data performed better than the model trained
on the unscaled data.

\subsubsection*{Notes}
\begin{itemize}
    \item \textit{Paraphrase parameters better}
\end{itemize}

\begin{table}[H]
    \begin{tcolorbox}[arc=0pt,boxrule=0.5pt]
        \sisetup{group-minimum-digits = 4}
        \centering
        \label{tab:hyperparameters_svr}
        \begin{tabular}{llp{9cm}}
            \toprule
            \thead{\textbf{Hyperparameter}} & \thead{\textbf{Value}} &
            \thead{\textbf{Description}}
            \\
            %  \unit{(Kcal\per\mole)\squared}}} & \thead{RMSD l.b.} &
            %  \thead{RMSD u.b.}  \\
            \toprule
            kernel & rbf & Kernel type used in the algorithm.
            \\
            \hdashline
            degree & 1 & Degree of polynomial kernel function.
            \\
            \hdashline
            gamma & 0.1 & Kernel coefficient for rbf, poly and sigmoid kernels.
            \\
            \hdashline
            C & 4000 & "Regularization parameter. The strength of the
            regularization is inversely proportional to C" \\
            \hdashline
            epsilon & 0.001 & "Epsilon in the epsilon-SVR
            model." \\
            \bottomrule
        \end{tabular}
        \caption{Hyperparameters of the Suport Vector Regressor.}
    \end{tcolorbox}
\end{table}


\section{Structure of The Code}\label{sec:structure-of-the-code}

\begin{itemize}
    \item \textit{The code is available on GitHub: \url{www.github.com/...}}
    \item \textit{TODO: Short explanation what to do to reproduce the results.}
\end{itemize}

% \paragraph{Bend deduction:}
% Measuring the bend deduction is more complex. After a metal sheet is bent,
% it is hard to
% measure the flat pattern
% length because the material is malformed at the bent.
% As a result, the neutral axis is not in the center of the sheet and hard to
% measure, but it can
% be calculated using
% different approaches. %quelle und ausführlicher und grunlagen teil
% There are multiple ways to measure the bend deduction described earlier. 
% % K-Faktor Muss noch in theorie teil 
% In this setup, the method described in the DIN6395 was used. This method
% uses a k-factor which
% is an approximated
% value and therefore and therefore it can be inaccurate.
% (Equation~\ref{eq:kfactor}). % cite DIN
% norm

% \begin{equation}\label{eq:kfactor}
%     k=0.65+\frac{1}{2}\log{\frac{r}{s}}
% \end{equation}

% \begin{figure}[ht!] % supposedly places it here ...
% 	\centering
% 	\includegraphics[width=0.5\linewidth]{k-factor}
% 	\caption[Graphical representation of the correction factor]{Graphical
% representation of the
% correction factor.}
% 	\label{fig:test1}
% \end{figure}

% The DIN 6935 used the formula for the stretched length, $length=a+b+v$
% where \textit{a} and
% \textit{b} are the side
% lengths of the sheet and \textit{v} is a correction value for the deduction
% . \cite{din6935}
% The stretched length is measured different depending on the bending angle.

% \paragraph{Opening angle $\beta 0^\circ$ to $90^\circ$} 
% For opening angles between $0^\circ$and $90^\circ$ the side lengths
% \textit{a} and \textit{b}
% are dimensioned from
% the tangent of the bend to the edge.
% To calculate the compensation value \textit{v} (Equation~\ref{eq:v1}) is used
% \cite{din6935}.

% \begin{equation}\label{eq:v1}
%         v=\pi*(\frac{180^\circ-}{180^\circ})*(r+\frac{s}{2}*k)-2(r+s)
% \end{equation}

% \begin{figure}[H]
% 	\centering
% 	\includegraphics[width=0.5\linewidth]{bending-angle-90}
% 	\caption[Opening angles $\beta 0^\circ$ to $90^\circ$]{Opening angles
% $\beta 0^\circ$ to
% $90^\circ$ \cite{din6935}}
% 	\label{fig:v1-image}
% \end{figure}

% \paragraph{Bending angle $\beta90^\circ$ to $165^\circ$}
% (Equation~\ref{eq:v1})
% For opening angles between $90^\circ$ and $165^\circ$ the side lengths
% \textit{a} and
% \textit{b} are dimensioned
% from the apex to the edge.
% To calculate the compensation value \textit{v} (Equation~\ref{eq:v1}) is
% used.
% \cite{din6935}

% \begin{equation}\label{eq:v2}
%     v=\pi*(\frac{180^\circ-}{180^\circ})*(r+\frac{s}{2}*k)-2(r+s)
% +\tan{\frac{180^\circ-\beta}{2}}
% \end{equation}

% \begin{figure}[!ht]
% 	\centering
% 	\includegraphics[width=0.5\linewidth]{bending-angle-165}
% 	\caption[Opening angles $\beta90^\circ$ to $165^\circ$]{Opening angles
% $\beta90^\circ$ to
% $165^\circ$
% \cite{din6935}}
% 	\label{fig:v2-image}
% \end{figure}

% For opening angles between $165^\circ$ and $180^\circ$ the compensation
% value \textit{v} is 0.
% The values for v
% would be negligibly small. \cite{din6935} The side lengths \textit{a} and
% \textit{b} where
% measured using the
% software \textit{ImageJ}.

% Edge cracking is not measured for now because the steel used has no
% high-strength and with
% machine in usage it was
% not possible to create edge cracking.

% \begin{figure}[!ht]
% 	\centering
% 	\includegraphics[width=0.5\linewidth]{example-image}
% 	\caption[Screenshot ImageJ]{Screenshot ImageJ}
% \end{figure}
% 	\label{fig:imagej-screenshot}