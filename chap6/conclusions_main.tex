\chapter{Conclusions}\label{ch:conclusions}


\section{Summary of the main findings}\label{subsec:summary-of-the-main-findings}
This study investigated the use of Machine Learning \ac{ML} models to accurately predict spring back in sheet metal
air-bending data, determin the best-performing models, and explore the potential of interpretability in ML models in
providing insights into the factors that contribute to spring back.

The results demonstrated that ML models, particularly the Extra Tree and Support Vector Machine models can accurately
predict the spring back, outperforming traditional trial-and-error methods.
The Extra Tree model showed the best overall performance across the six Design Principles (DPs), highlighting the
importance to choose the right model depending on the application.


\section{Answering the research question}\label{sec:answering-the-research-question}
In this section the findings are summarized to answer the research questions of this study.

\subsection{Research Question 1: Can Machine Learning Models Accurately Predict Spring Back in Sheet Metal Forming?}
\label{subsec:research-question-1:-can-machine-learning-models-accurately-predict-spring-back-in-sheet-metal-forming?}

The findings of this study support Hypothesis 1 and demontrate that \ac{ML} models cann accurately predict the spring
back in sheet metal bending.
The best performing models, Extra Trees and Support Vector Machine, achieved an RMSE of 0.15 mm, which is acceptable
for many applications, considering the range of spring back values (0.5 mm to 3.1 mm).
The \ac{ML} models outperform traditional trial-and-error methods, such as technology tables, by providing accurate
predictions for a broader range of samples and bends.

\subsection{Research Question 2: Which ML Models or Combinations of Models Yield the Best Performance?}
\label{subsec:research-question-2:-which-ml-models-or-combinations-of-models-yield-the-best-performance?}
The study's result answer Hypothesis 2, showing that specific \ac{ML} models perform better based on the six Design
Principles used.
The Extra Trees model demonstrated the best balance between bias and variance and robust performance when exposed to
missing data na noise.
The gradient Boosting model displayer the highest stability, while the Extra Rees and Random Forest models exhibited the
the best resource utilization.
The findings highlight the importance of selecting suitable \ac{ML} models for specific applications to achieve the
best performance.

\subsection{Research Question 3: Can ML Models with High Interpretability Provide Valuable Insights into the Factors
Contributing to Spring Back?}
\label{subsec:research-question-3:-can-ml-models-with-high-interpretability-provide
-valuable-insights-into-the-factors-
contributing-to-spring-back?}
This study's findings partially support Hypothesis 3, indicating that interpretability methods can provide valuable
insights into the factors that contribute to spring back.
The thickness of the metal sheet was found to have the most significant influence on the spring back, but also punch
penetration and die opening played important roles.
Model-agnostic interpretability methods revealed non-linear dependencies and interactions between features, which can
be valuable for refining the models and developing more effective compensations strategies.

Furthermore, it is important to note that certain models which are considered as interpretable such as decision trees
or linear models could not be used in this study due to their insufficient performance.
This suggest that a balance between interpretability and performance is necessary for gaining valuable insights into
the factors affecting spring back.j


\section{Classification in the state of research}\label{sec:classification-in-the-state-of-research}
This study contributes to the existing body of research on the applicatoins of ML models to metal forming processes.
By using data that was generated in the real-world and a more comprehensive evaluation framework based on six Design
Principles, the study could provide a foundations for future research in this field.
The findings of this study not only apply on the spring back predicion but also have broad implications for other
evaluations of ML models in other domain.


\section{Outlook for future research}\label{subsec:outlook-for-future-research}


Future research can build upon the findings of this study by exploring several ways.

\begin{itemize}
    \item Investigating the use of more complex neural networks to improve the prediction accuracy of ML models in
    \item spring back
    prediction.
    \item Examining the performance of ensemble models and their potential in delivering better performance compared to
    individual models.
    \item Collecting and incorporating more data for high spring back cases to enhance the performance of ML models in
    predicting extreme cases.
    \item Developing standardized evaluation frameworks for ML models in various fields, drawing inspiration from the
    \item Design
    Principles used in this study.
    \item Investigating the use of other interpretable ML models or model-agnostic methods to gain deeper insights
    \item into the
    factors affecting spring back and the underlying physical processes in sheet metal forming.
\end{itemize}

By addressing these areas, future research can further advance the understanding and application of ML models in
predicting spring back and other complex processes.


\section{Conclusion}\label{sec:conclusion}
This study demonstrated the potential of \ac{ML} models to accurately predict spring back in sheet in the air
-bending metal forming process using real-world data.
The results support the three hypotheses posed, showing that \ac{ML} models can outperform traditional methods and
that specific models or combinations of models yield better romance.
Also that that interpretability methods can be used to provide valuable insights into the factors contributing to
spring back.
The findings of this study can be used to guide the development of \ac{ML} models for other complex processes and
provide a foundation for future research in this field.
