%% For tips on how to write a great abstract, have a look at
%%	-	https://www.cdc.gov/stdconference/2018/How-to-Write-an-Abstract_v4.pdf (presentation, start here)
%%	-	https://users.ece.cmu.edu/~koopman/essays/abstract.html
%%	-	https://search.proquest.com/docview/1417403858
%%  - 	https://www.sciencedirect.com/science/article/pii/S037837821830402X

%\begin{abstract}
\section*{Abstract}
\noindent Sheet metal bending is a widely ued manufacturing process in various industries such as automotive,
aerospace and construction.
One of the challenges in this process is the accurate prediction and compensation of spring back, which occurs
due the elastic recovery of the material after bending.
Inaccurate predictions of spring back can lead do fitting issues, increased costs and reduced product quality.
This study aims to research the potential of Machine Learning (ML) models to predict the spring back in sheet
metal bending using real-world air-bending data.

Three hypotheses were formulated and tested in this study.
First, it was hypothesized that ML models can accurately predict spring back, outperforming traditional trial-and-error methods.
Second, it was hypothesised that specific ML models or combinations of models would yield better performance
based on six Design Principles.
Third, it was hypothesized that ML models with high interpretability can provide valuable insights into the
factors contributing to spring back.

The results of this study support all three hypotheses, demonstrating the effectiveness of ML models in predicting the
spring back in sheet metal bending.
The Multi-Layer-Perceptron and Support Vector Machine models achieved the best performance with a root mean squared
error (RMSE) of 0.2 mm. Multiple models statisfy the highest accuracy level of the ISO 2768.
Additionally non-linear interactions between the different features in the dataset where found and analysed within
the dataset.
The findings also highlight the importance of selecting suitable ML models for specific applications.


In conclusion, this study demonstrates the potential of ML models to accurately predict and compensate for spring
back in sheet metal forming processes, leading to improved product quality and reduced manufacturing costs.
The findings have implications for both the understanding of spring back phenomena and the development of more
effective compensation strategies in the sheet metal forming industry.
%\end{abstract}