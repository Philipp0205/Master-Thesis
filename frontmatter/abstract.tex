%% For tips on how to write a great abstract, have a look at
%%	-	https://www.cdc.gov/stdconference/2018/How-to-Write-an-Abstract_v4.pdf (presentation, start here)
%%	-	https://users.ece.cmu.edu/~koopman/essays/abstract.html
%%	-	https://search.proquest.com/docview/1417403858
%%  - 	https://www.sciencedirect.com/science/article/pii/S037837821830402X

\begin{abstract}
    Sheet metal bending is a widely ued manufacturing process in various industries such as automotove, aerospace and
    construction.
    One of the challenges in this process is the accurate prediction and compensation of spring back, which occurs
    due the elastic recovery of the material after bending.
    Inaccurate predictions of spring back can lear do fitting issued, increased costs and reduced product quality.
    This study aims to research the potential of Machine Learning (ML) models to predict the sprin back in sheet
    metal bending using real-world air-bending data.

    Three hypotheses were formulated and tested in this study.
    First, it was hypothesized that ML models can accurately predict spring back, outperforming traditional trial-and
    -error methods.
    Second, it was hypothesised that specific ML models or combinations of models would yield better performance
    based on six Design Principles.
    Third, it was hypothesized that ML models with high interpretability can provide valuable insights into the
    factors contributing to spring back.

    The results of this study support all three hypotheses, demonstrating the effectiveness of ML models in predicting
    spring back in sheet metal bending.
    The Extra Trees and Support Vector Machine models achieved the best performance with a root mean squared error
    (RMSE) of 0.15 mm.
    The findings also highlight the importance of selecting suitable ML models ofr specific applications and the
    valuable insights that can be gained from models with high interpretability.

    In conclusion, this study demonstrates the potential of ML models to accurately predict and compensate for spring
    back in sheet metal forming processes, leading to improved product quality and reduced manufacturing costs.
    The findings have implications for both the understanding of spring back phenomena and the development of more
    effective compensation strategies in the sheet metal forming industry.
\end{abstract}